\documentclass[12pt]{article}
\usepackage{setspace}
\begin{document}
\doublespacing
%\singlespacing
Perl was created by Larry Wall in 1987, and was based on C, sh, and awk.\cite{timeline} Wall created Perl after he ran into a problem that would have been hard to solve with shell or awk.\cite{interview} Perl is an interpreted language and was meant to be easy to use and efficient rather than elegant.\cite{manpage} Perl was designed to quickly scan text files avoid as many memory limitations as possible.\cite{manpage}\\ 

Wall wanted to give his programming language a positive name. After much thought on the subject, even going as far as considering naming it after his wife, Gloria, Wall finally decided to name it after the gemstone pearl. Wall later changed the name from "pearl" to "perl" after hearing about a graphics language also called "pearl." Due to limitations with UNIX, Perl was originally written with a lowercase "p," but was later changed to an uppercase "P" for easier distiction between the language and the program.\cite{interview}\\

Over the years, Perl has earned many nicknames describing its usefulness. One of the earliest was created by Wall himselff, making Perl the unofficial acronym "Practical Extraction and Report Language."\cite{interview} Other nicknames have included "The duct tape of the internet\cite {internet} and "The Swiss Army chainsaw."\cite{chainsaw} \\

Perl is known as "The duct tape of the internet" because of its ability to "glue" different programs together, allowing one program to input into another. Because of this, Perl became an ideal choice when working with The Common Gateway Interface, a tool used to link a server, program and web page together.\cite{internet} As a result, many modern websites use Perl in some way, including IMDB, Amazon, BBC, and Yahoo\cite{list}\cite{joy}\\

Perl Is also known as "the Swiss Army chainsaw" because of its power and adaptability. Perl is so flexible that programmers can often form their own style. this is possible because problems in Perl usually have multiple solutions.cite{chainsaw}\\

unlike many C-based languages, Perl doesn't use ints, strings, doubles, etc. Instead, Perl uses scalers (\$), arrays (@), and hashes (\%). each symbol is included every time the variable is used and indicates what kind of variable it is. a scaler is a basic variable and can be an int, string, double, etc. An array is the same as arrays in C-based languages, and a hash is like an array whose items are linked to a specific keyword.\cite{chainsaw}\\

to summerize, Perl is a language created by combining the aspects of many other languages. This combination of languages lead to it being flexible and powerful, but very different tan other languages. Finally, much of Perl's power comes from its ability to combine programs.
%Perl has a large variaty of uses. Perl has many libraries available for multitasking, games, GUIs, and Databases. Perl is also commonly used to create websites. Websites that use Perl include IMDB, Amazocn, BBC, and Yahoo\cite{list}\cite{joy}
\begin{thebibliography}{9}
\bibitem{interview}
Richardson, Marjorie. 
"Larry Wall, the Guru of Perl." 
\textit{Linux Journal}. 
N.p. 
1. May. 1999. 
18. Sept. 2017.

\bibitem{timeline}
Ashton, Elaine.
"The Timeline of Perl and its Culture."
\textit{history.pearl.org}. 
N.p. 
N.d.
18. Sept. 2017.

\bibitem{manpage}
perl - Practical Extraction and Report Language
\textit{Perl Man Page}. 
N.p. 
N.d.
18. Sept. 2017.

\bibitem{list}
5 major websites that use Perl
\textit{BIP Blog}. 
N.p. 
2. Jun. 2016.
18. Sept. 2017.

\bibitem{chainsaw}
Sheppard, Doug
Beginner's Introduction to Perl
\textit{Perl.com}. 
N.p. 
16. Oct. 2000.
20. Sept. 2017.

\bibitem{joy}
Leonard, Andrew
The joy of Perl
\textit{Salon}. 
N.p. 
13. Oct. 1998.
18. Sept. 2017.

\bibitem{internet}
O'Reilly, Tim
The Importance of Perl
\textit{ONLAMP}. 
N.p. 
N.d.
20. Sept. 2017.
\end{thebibliography}
\end{document} 